\documentclass{beamer}
%%%%%%%%%%%%%%%%%%%%%%%%%%%%%%%  Packages  %%%%%%%%%%%%%
\usepackage{amsmath} 
\usepackage{mathtools}
\usepackage{physics}
\usepackage{amssymb}
\usepackage{mathptmx}
\usepackage{array}
  
%%%%%%%%% FIGurES %%%%%%%%%%%%%%%%%%%%%%%%
\usepackage{textcomp}
\usepackage{graphicx}
\usepackage{caption} 
\usepackage{subcaption}
\usepackage{scrextend}
\usepackage{rotating}
\usepackage{float}
\usepackage{hyperref}
\hypersetup{colorlinks=true, citecolor=blue, linkcolor=blue}
\renewcommand{\equationautorefname}{Eq.}%
\renewcommand{\figureautorefname}{Fig.}%
 
%%%%%%%%%%%% LaNgUaGe %%%%%%%%%%%%%%%%%%
\usepackage{verbatim}
\usepackage{natbib}
%\usepackage{qcircuit}
\usepackage{wrapfig}

\usepackage[utf8]{inputenc}
\usetheme{PaloAlto}

\title{Version control using Git and Plotting Tutorial}
\author{Oliver Thomas}
\institute{Quantum Engineering CDT \\ University of Bristol}
\date{\today}

\begin{document}

\frame{\titlepage}

% slide 1
\begin{frame}
\frametitle{Why you should use version control}
\begin{itemize}
\item Collaborative work  
\item Working remotely
\end{itemize}
\end{frame}

%slide 2
\begin{frame}
\frametitle{What is Git?}
\begin{itemize}
\item  
\item
\end{itemize}
\end{frame}

%slide 3
\begin{frame}
\frametitle{Making a repository}
\begin{itemize}
\item You can do this online on the Github website   
\item Create a new repository
\item Then click clone to get the url, open git on your computer and type: 
\begin{verbatim}
git clone url
\end{verbatim}
\item 
\end{itemize}
\end{frame}

%slide 4
\begin{frame}
\frametitle{Basic Git commands}
\begin{itemize}
	\item \begin{verbatim} git pull \end{verbatim} 
	\item \begin{verbatim} git add \end{verbatim}
	\item \begin{verbatim} git commit \end{verbatim}
	\item \begin{verbatim} git push \end{verbatim}
\end{itemize}
\end{frame}

%slide 5
\begin{frame}
\frametitle{What} 
\begin{itemize}
\item 
\item
\end{itemize}
\end{frame}

%slide 6
\begin{frame}
\frametitle{}
\end{frame}

\end{document}
